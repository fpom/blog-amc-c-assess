\expandafter\def\csname cpp:swap\endcsname{1}
\expandafter\def\csname cpp:PARAM(x,y)\endcsname{y, x}
\expandafter\def\csname cpp:comp\endcsname{max}
\expandafter\def\csname cpp:INIT\endcsname{0}
\expandafter\def\csname cpp:PROCESS(c,v)\endcsname{c = v}
\expandafter\def\csname cpp:NOT\endcsname{0}
\expandafter\def\csname cpp:COND(f,c,v)\endcsname{(v > c) && f}
\expandafter\def\csname cpp:FALSE\endcsname{0}
\expandafter\def\csname cpp:TRUE\endcsname{1}
\expandafter\def\csname cpp:join\endcsname{and}
\expandafter\def\csname cpp:FORALLIN\endcsname{roar}
\expandafter\def\csname cpp:A\endcsname{gerbil}
\expandafter\def\csname cpp:B\endcsname{filmed}
\expandafter\def\csname cpp:C\endcsname{franc}
\expandafter\def\csname cpp:I\endcsname{colors}
\expandafter\def\csname cpp:J\endcsname{bathing}
\expandafter\def\csname cpp:F\endcsname{side}
\expandafter\def\csname cpp:U\endcsname{wood}
\expandafter\def\csname cpp:TABU_LEN\endcsname{6}
\expandafter\def\csname cpp:TABU_VAL\endcsname{[6, 7, 7, 10, 1, 5]}
\expandafter\def\csname cpp:TABU\endcsname{{.len=6, .val=malloc(6 * sizeof(int))}}
\expandafter\def\csname cpp:V\endcsname{maces}
\expandafter\def\csname cpp:TABV_LEN\endcsname{6}
\expandafter\def\csname cpp:TABV_VAL\endcsname{[5, 1, 2, 1, 7, 1]}
\expandafter\def\csname cpp:TABV\endcsname{{.len=6, .val=malloc(6 * sizeof(int))}}

\def\C#1{\csname cpp:#1\endcsname}

\element{COMP7}{
  \begin{questionmult}{C1}
    What does function \C{FORALLIN} computes?
    \AMCBoxedAnswers
    \begin{reponses}
      \wrongchoice{the smallest value that is in \texttt{gerbil} but not in \texttt{filmed}}
      \wrongchoice{the largest value that is in \texttt{gerbil} but not in \texttt{filmed}}
      \wrongchoice{the sum of the values that are in \texttt{gerbil} but not in \texttt{filmed}}
      \wrongchoice{the number of values that are in \texttt{gerbil} but not in \texttt{filmed}}
      \wrongchoice{the smallest value that is both in \texttt{gerbil} and in \texttt{filmed}}
      \wrongchoice{the largest value that is both in \texttt{gerbil} and in \texttt{filmed}}
      \wrongchoice{the sum of the values that are both in \texttt{gerbil} and in \texttt{filmed}}
      \wrongchoice{the number of values that are both in \texttt{gerbil} and in \texttt{filmed}}
      \wrongchoice{the smallest value that is in \texttt{filmed} but not in \texttt{gerbil}}
      \wrongchoice{the largest value that is in \texttt{filmed} but not in \texttt{gerbil}}
      \wrongchoice{the sum of the values that are in \texttt{filmed} but not in \texttt{gerbil}}
      \wrongchoice{the number of values that are in \texttt{filmed} but not in \texttt{gerbil}}
      \wrongchoice{the smallest value that is both in \texttt{filmed} and in \texttt{gerbil}}
      \correctchoice{the largest value that is both in \texttt{filmed} and in \texttt{gerbil}}
      \wrongchoice{the sum of the values that are both in \texttt{filmed} and in \texttt{gerbil}}
      \wrongchoice{the number of values that are both in \texttt{filmed} and in \texttt{gerbil}}
    \end{reponses}
  \end{questionmult}
  \begin{questionmult}{C2}
    What does the program prints?
    \begin{multicols}{4}
      \AMCBoxedAnswers
      \begin{reponses}
        \wrongchoice{1}
        \wrongchoice{2}
        \wrongchoice{4}
        \wrongchoice{5}
        \wrongchoice{6}
        \correctchoice{7}
        \wrongchoice{10}
        \wrongchoice{15}
        \wrongchoice{16}
        \wrongchoice{20}
        \wrongchoice{21}
        \wrongchoice{29}
      \end{reponses}
    \end{multicols}
  \end{questionmult}
  \begin{questionmult}{C3}
    What does the program prints?
    \AMCnumericChoices{7}{digits=3,vertical=true,scoreexact=1,scoreapprox=0,borderwidth=1pt,sign=false}
  \end{questionmult}
}

