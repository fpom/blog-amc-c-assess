\expandafter\def\csname cpp:swap\endcsname{0}
\expandafter\def\csname cpp:PARAM(x,y)\endcsname{x, y}
\expandafter\def\csname cpp:comp\endcsname{min}
\expandafter\def\csname cpp:INIT\endcsname{999}
\expandafter\def\csname cpp:PROCESS(c,v)\endcsname{c = v}
\expandafter\def\csname cpp:NOT\endcsname{1}
\expandafter\def\csname cpp:COND(f,c,v)\endcsname{(!f) && (v < c)}
\expandafter\def\csname cpp:FALSE\endcsname{0}
\expandafter\def\csname cpp:TRUE\endcsname{1}
\expandafter\def\csname cpp:join\endcsname{sub}
\expandafter\def\csname cpp:FORALLIN\endcsname{cot}
\expandafter\def\csname cpp:A\endcsname{rough}
\expandafter\def\csname cpp:B\endcsname{returns}
\expandafter\def\csname cpp:C\endcsname{bleeder}
\expandafter\def\csname cpp:I\endcsname{hues}
\expandafter\def\csname cpp:J\endcsname{chaff}
\expandafter\def\csname cpp:F\endcsname{endive}
\expandafter\def\csname cpp:U\endcsname{buyers}
\expandafter\def\csname cpp:TABU_LEN\endcsname{6}
\expandafter\def\csname cpp:TABU_VAL\endcsname{[6, 3, 1, 9, 9, 4]}
\expandafter\def\csname cpp:TABU\endcsname{{.len=6, .val=malloc(6 * sizeof(int))}}
\expandafter\def\csname cpp:V\endcsname{wimpled}
\expandafter\def\csname cpp:TABV_LEN\endcsname{5}
\expandafter\def\csname cpp:TABV_VAL\endcsname{[6, 2, 2, 1, 7]}
\expandafter\def\csname cpp:TABV\endcsname{{.len=5, .val=malloc(5 * sizeof(int))}}

\def\C#1{\csname cpp:#1\endcsname}

\element{COMP5}{
  \begin{questionmult}{C1}
    What does function \C{FORALLIN} computes?
    \AMCBoxedAnswers
    \begin{reponses}
      \correctchoice{the smallest value that is in \texttt{rough} but not in \texttt{returns}}
      \wrongchoice{the largest value that is in \texttt{rough} but not in \texttt{returns}}
      \wrongchoice{the sum of the values that are in \texttt{rough} but not in \texttt{returns}}
      \wrongchoice{the number of values that are in \texttt{rough} but not in \texttt{returns}}
      \wrongchoice{the smallest value that is both in \texttt{rough} and in \texttt{returns}}
      \wrongchoice{the largest value that is both in \texttt{rough} and in \texttt{returns}}
      \wrongchoice{the sum of the values that are both in \texttt{rough} and in \texttt{returns}}
      \wrongchoice{the number of values that are both in \texttt{rough} and in \texttt{returns}}
      \wrongchoice{the smallest value that is in \texttt{returns} but not in \texttt{rough}}
      \wrongchoice{the largest value that is in \texttt{returns} but not in \texttt{rough}}
      \wrongchoice{the sum of the values that are in \texttt{returns} but not in \texttt{rough}}
      \wrongchoice{the number of values that are in \texttt{returns} but not in \texttt{rough}}
      \wrongchoice{the smallest value that is both in \texttt{returns} and in \texttt{rough}}
      \wrongchoice{the largest value that is both in \texttt{returns} and in \texttt{rough}}
      \wrongchoice{the sum of the values that are both in \texttt{returns} and in \texttt{rough}}
      \wrongchoice{the number of values that are both in \texttt{returns} and in \texttt{rough}}
    \end{reponses}
  \end{questionmult}
  \begin{questionmult}{C2}
    What does the program prints?
    \begin{multicols}{4}
      \AMCBoxedAnswers
      \begin{reponses}
        \wrongchoice{0}
        \wrongchoice{1}
        \wrongchoice{2}
        \correctchoice{3}
        \wrongchoice{4}
        \wrongchoice{6}
        \wrongchoice{7}
        \wrongchoice{9}
        \wrongchoice{11}
        \wrongchoice{21}
        \wrongchoice{24}
        \wrongchoice{25}
      \end{reponses}
    \end{multicols}
  \end{questionmult}
  \begin{questionmult}{C3}
    What does the program prints?
    \AMCnumericChoices{3}{digits=3,vertical=true,scoreexact=1,scoreapprox=0,borderwidth=1pt,sign=false}
  \end{questionmult}
}

