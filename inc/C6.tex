\expandafter\def\csname cpp:swap\endcsname{1}
\expandafter\def\csname cpp:PARAM(x,y)\endcsname{y, x}
\expandafter\def\csname cpp:comp\endcsname{min}
\expandafter\def\csname cpp:INIT\endcsname{999}
\expandafter\def\csname cpp:PROCESS(c,v)\endcsname{c = v}
\expandafter\def\csname cpp:NOT\endcsname{0}
\expandafter\def\csname cpp:COND(f,c,v)\endcsname{f && (c > v)}
\expandafter\def\csname cpp:FALSE\endcsname{0}
\expandafter\def\csname cpp:TRUE\endcsname{1}
\expandafter\def\csname cpp:join\endcsname{and}
\expandafter\def\csname cpp:FORALLIN\endcsname{states}
\expandafter\def\csname cpp:A\endcsname{atom}
\expandafter\def\csname cpp:B\endcsname{insult}
\expandafter\def\csname cpp:C\endcsname{linked}
\expandafter\def\csname cpp:I\endcsname{itchier}
\expandafter\def\csname cpp:J\endcsname{chassis}
\expandafter\def\csname cpp:F\endcsname{bathe}
\expandafter\def\csname cpp:U\endcsname{callow}
\expandafter\def\csname cpp:TABU_LEN\endcsname{10}
\expandafter\def\csname cpp:TABU_VAL\endcsname{[2, 10, 9, 8, 7, 1, 4, 7, 1, 1]}
\expandafter\def\csname cpp:TABU\endcsname{{.len=10, .val=malloc(10 * sizeof(int))}}
\expandafter\def\csname cpp:V\endcsname{quires}
\expandafter\def\csname cpp:TABV_LEN\endcsname{9}
\expandafter\def\csname cpp:TABV_VAL\endcsname{[7, 1, 8, 1, 5, 5, 7, 9, 6]}
\expandafter\def\csname cpp:TABV\endcsname{{.len=9, .val=malloc(9 * sizeof(int))}}

\def\C#1{\csname cpp:#1\endcsname}

\element{COMP6}{
  \begin{questionmult}{C1}
    What does function \C{FORALLIN} computes?
    \AMCBoxedAnswers
    \begin{reponses}
      \wrongchoice{the smallest value that is in \texttt{atom} but not in \texttt{insult}}
      \wrongchoice{the largest value that is in \texttt{atom} but not in \texttt{insult}}
      \wrongchoice{the sum of the values that are in \texttt{atom} but not in \texttt{insult}}
      \wrongchoice{the number of values that are in \texttt{atom} but not in \texttt{insult}}
      \wrongchoice{the smallest value that is both in \texttt{atom} and in \texttt{insult}}
      \wrongchoice{the largest value that is both in \texttt{atom} and in \texttt{insult}}
      \wrongchoice{the sum of the values that are both in \texttt{atom} and in \texttt{insult}}
      \wrongchoice{the number of values that are both in \texttt{atom} and in \texttt{insult}}
      \wrongchoice{the smallest value that is in \texttt{insult} but not in \texttt{atom}}
      \wrongchoice{the largest value that is in \texttt{insult} but not in \texttt{atom}}
      \wrongchoice{the sum of the values that are in \texttt{insult} but not in \texttt{atom}}
      \wrongchoice{the number of values that are in \texttt{insult} but not in \texttt{atom}}
      \correctchoice{the smallest value that is both in \texttt{insult} and in \texttt{atom}}
      \wrongchoice{the largest value that is both in \texttt{insult} and in \texttt{atom}}
      \wrongchoice{the sum of the values that are both in \texttt{insult} and in \texttt{atom}}
      \wrongchoice{the number of values that are both in \texttt{insult} and in \texttt{atom}}
    \end{reponses}
  \end{questionmult}
  \begin{questionmult}{C2}
    What does the program prints?
    \begin{multicols}{4}
      \AMCBoxedAnswers
      \begin{reponses}
        \correctchoice{1}
        \wrongchoice{2}
        \wrongchoice{3}
        \wrongchoice{34}
        \wrongchoice{5}
        \wrongchoice{6}
        \wrongchoice{7}
        \wrongchoice{33}
        \wrongchoice{9}
        \wrongchoice{10}
        \wrongchoice{13}
        \wrongchoice{16}
        \wrongchoice{21}
        \wrongchoice{29}
      \end{reponses}
    \end{multicols}
  \end{questionmult}
  \begin{questionmult}{C3}
    What does the program prints?
    \AMCnumericChoices{1}{digits=3,vertical=true,scoreexact=1,scoreapprox=0,borderwidth=1pt,sign=false}
  \end{questionmult}
}

