\expandafter\def\csname cpp:swap\endcsname{1}
\expandafter\def\csname cpp:PARAM(x,y)\endcsname{y, x}
\expandafter\def\csname cpp:comp\endcsname{sum}
\expandafter\def\csname cpp:INIT\endcsname{0}
\expandafter\def\csname cpp:PROCESS(c,v)\endcsname{c += v}
\expandafter\def\csname cpp:NOT\endcsname{1}
\expandafter\def\csname cpp:COND(f,c,v)\endcsname{!f}
\expandafter\def\csname cpp:FALSE\endcsname{1}
\expandafter\def\csname cpp:TRUE\endcsname{0}
\expandafter\def\csname cpp:join\endcsname{and}
\expandafter\def\csname cpp:FORALLIN\endcsname{sired}
\expandafter\def\csname cpp:A\endcsname{same}
\expandafter\def\csname cpp:B\endcsname{phlox}
\expandafter\def\csname cpp:C\endcsname{hauls}
\expandafter\def\csname cpp:I\endcsname{meaner}
\expandafter\def\csname cpp:J\endcsname{clawed}
\expandafter\def\csname cpp:F\endcsname{wattle}
\expandafter\def\csname cpp:U\endcsname{scruffy}
\expandafter\def\csname cpp:TABU_LEN\endcsname{10}
\expandafter\def\csname cpp:TABU_VAL\endcsname{[8, 7, 5, 9, 6, 5, 5, 3, 8, 1]}
\expandafter\def\csname cpp:TABU\endcsname{{.len=10, .val=malloc(10 * sizeof(int))}}
\expandafter\def\csname cpp:V\endcsname{fiddly}
\expandafter\def\csname cpp:TABV_LEN\endcsname{10}
\expandafter\def\csname cpp:TABV_VAL\endcsname{[4, 10, 1, 3, 8, 9, 8, 7, 4, 10]}
\expandafter\def\csname cpp:TABV\endcsname{{.len=10, .val=malloc(10 * sizeof(int))}}

\def\C#1{\csname cpp:#1\endcsname}

\element{COMP4}{
  \begin{questionmult}{C1}
    What does function \C{FORALLIN} computes?
    \AMCBoxedAnswers
    \begin{reponses}
      \wrongchoice{the smallest value that is in \texttt{same} but not in \texttt{phlox}}
      \wrongchoice{the largest value that is in \texttt{same} but not in \texttt{phlox}}
      \wrongchoice{the sum of the values that are in \texttt{same} but not in \texttt{phlox}}
      \wrongchoice{the number of values that are in \texttt{same} but not in \texttt{phlox}}
      \wrongchoice{the smallest value that is both in \texttt{same} and in \texttt{phlox}}
      \wrongchoice{the largest value that is both in \texttt{same} and in \texttt{phlox}}
      \wrongchoice{the sum of the values that are both in \texttt{same} and in \texttt{phlox}}
      \wrongchoice{the number of values that are both in \texttt{same} and in \texttt{phlox}}
      \wrongchoice{the smallest value that is in \texttt{phlox} but not in \texttt{same}}
      \wrongchoice{the largest value that is in \texttt{phlox} but not in \texttt{same}}
      \wrongchoice{the sum of the values that are in \texttt{phlox} but not in \texttt{same}}
      \wrongchoice{the number of values that are in \texttt{phlox} but not in \texttt{same}}
      \wrongchoice{the smallest value that is both in \texttt{phlox} and in \texttt{same}}
      \wrongchoice{the largest value that is both in \texttt{phlox} and in \texttt{same}}
      \correctchoice{the sum of the values that are both in \texttt{phlox} and in \texttt{same}}
      \wrongchoice{the number of values that are both in \texttt{phlox} and in \texttt{same}}
    \end{reponses}
  \end{questionmult}
  \begin{questionmult}{C2}
    What does the program prints?
    \begin{multicols}{4}
      \AMCBoxedAnswers
      \begin{reponses}
        \wrongchoice{1}
        \wrongchoice{4}
        \wrongchoice{5}
        \wrongchoice{6}
        \correctchoice{36}
        \wrongchoice{9}
        \wrongchoice{10}
        \wrongchoice{14}
        \wrongchoice{19}
        \wrongchoice{20}
        \wrongchoice{21}
        \wrongchoice{26}
        \wrongchoice{28}
      \end{reponses}
    \end{multicols}
  \end{questionmult}
  \begin{questionmult}{C3}
    What does the program prints?
    \AMCnumericChoices{36}{digits=3,vertical=true,scoreexact=1,scoreapprox=0,borderwidth=1pt,sign=false}
  \end{questionmult}
}

