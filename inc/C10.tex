\expandafter\def\csname cpp:swap\endcsname{0}
\expandafter\def\csname cpp:PARAM(x,y)\endcsname{x, y}
\expandafter\def\csname cpp:comp\endcsname{sum}
\expandafter\def\csname cpp:INIT\endcsname{0}
\expandafter\def\csname cpp:PROCESS(c,v)\endcsname{c += v}
\expandafter\def\csname cpp:NOT\endcsname{1}
\expandafter\def\csname cpp:COND(f,c,v)\endcsname{!f}
\expandafter\def\csname cpp:FALSE\endcsname{0}
\expandafter\def\csname cpp:TRUE\endcsname{1}
\expandafter\def\csname cpp:join\endcsname{sub}
\expandafter\def\csname cpp:FORALLIN\endcsname{gyrated}
\expandafter\def\csname cpp:A\endcsname{depute}
\expandafter\def\csname cpp:B\endcsname{corneas}
\expandafter\def\csname cpp:C\endcsname{discuss}
\expandafter\def\csname cpp:I\endcsname{cellars}
\expandafter\def\csname cpp:J\endcsname{alien}
\expandafter\def\csname cpp:F\endcsname{plush}
\expandafter\def\csname cpp:U\endcsname{charms}
\expandafter\def\csname cpp:TABU_LEN\endcsname{8}
\expandafter\def\csname cpp:TABU_VAL\endcsname{[4, 2, 4, 6, 4, 9, 4, 10]}
\expandafter\def\csname cpp:TABU\endcsname{{.len=8, .val=malloc(8 * sizeof(int))}}
\expandafter\def\csname cpp:V\endcsname{impels}
\expandafter\def\csname cpp:TABV_LEN\endcsname{6}
\expandafter\def\csname cpp:TABV_VAL\endcsname{[1, 2, 10, 4, 7, 8]}
\expandafter\def\csname cpp:TABV\endcsname{{.len=6, .val=malloc(6 * sizeof(int))}}

\def\C#1{\csname cpp:#1\endcsname}

\element{COMP10}{
  \begin{questionmult}{C1}
    What does function \C{FORALLIN} computes?
    \AMCBoxedAnswers
    \begin{reponses}
      \wrongchoice{the smallest value that is in \texttt{depute} but not in \texttt{corneas}}
      \wrongchoice{the largest value that is in \texttt{depute} but not in \texttt{corneas}}
      \correctchoice{the sum of the values that are in \texttt{depute} but not in \texttt{corneas}}
      \wrongchoice{the number of values that are in \texttt{depute} but not in \texttt{corneas}}
      \wrongchoice{the smallest value that is both in \texttt{depute} and in \texttt{corneas}}
      \wrongchoice{the largest value that is both in \texttt{depute} and in \texttt{corneas}}
      \wrongchoice{the sum of the values that are both in \texttt{depute} and in \texttt{corneas}}
      \wrongchoice{the number of values that are both in \texttt{depute} and in \texttt{corneas}}
      \wrongchoice{the smallest value that is in \texttt{corneas} but not in \texttt{depute}}
      \wrongchoice{the largest value that is in \texttt{corneas} but not in \texttt{depute}}
      \wrongchoice{the sum of the values that are in \texttt{corneas} but not in \texttt{depute}}
      \wrongchoice{the number of values that are in \texttt{corneas} but not in \texttt{depute}}
      \wrongchoice{the smallest value that is both in \texttt{corneas} and in \texttt{depute}}
      \wrongchoice{the largest value that is both in \texttt{corneas} and in \texttt{depute}}
      \wrongchoice{the sum of the values that are both in \texttt{corneas} and in \texttt{depute}}
      \wrongchoice{the number of values that are both in \texttt{corneas} and in \texttt{depute}}
    \end{reponses}
  \end{questionmult}
  \begin{questionmult}{C2}
    What does the program prints?
    \begin{multicols}{4}
      \AMCBoxedAnswers
      \begin{reponses}
        \wrongchoice{1}
        \wrongchoice{2}
        \wrongchoice{3}
        \wrongchoice{4}
        \wrongchoice{6}
        \wrongchoice{7}
        \wrongchoice{8}
        \wrongchoice{9}
        \wrongchoice{10}
        \wrongchoice{13}
        \correctchoice{15}
        \wrongchoice{16}
        \wrongchoice{18}
        \wrongchoice{28}
      \end{reponses}
    \end{multicols}
  \end{questionmult}
  \begin{questionmult}{C3}
    What does the program prints?
    \AMCnumericChoices{15}{digits=3,vertical=true,scoreexact=1,scoreapprox=0,borderwidth=1pt,sign=false}
  \end{questionmult}
}

