\expandafter\def\csname cpp:swap\endcsname{1}
\expandafter\def\csname cpp:PARAM(x,y)\endcsname{y, x}
\expandafter\def\csname cpp:comp\endcsname{max}
\expandafter\def\csname cpp:INIT\endcsname{0}
\expandafter\def\csname cpp:PROCESS(c,v)\endcsname{c = v}
\expandafter\def\csname cpp:NOT\endcsname{1}
\expandafter\def\csname cpp:COND(f,c,v)\endcsname{(!f) && (v > c)}
\expandafter\def\csname cpp:FALSE\endcsname{0}
\expandafter\def\csname cpp:TRUE\endcsname{1}
\expandafter\def\csname cpp:join\endcsname{sub}
\expandafter\def\csname cpp:FORALLIN\endcsname{junkie}
\expandafter\def\csname cpp:A\endcsname{news}
\expandafter\def\csname cpp:B\endcsname{dislike}
\expandafter\def\csname cpp:C\endcsname{undo}
\expandafter\def\csname cpp:I\endcsname{abode}
\expandafter\def\csname cpp:J\endcsname{severer}
\expandafter\def\csname cpp:F\endcsname{dizzy}
\expandafter\def\csname cpp:U\endcsname{skimpy}
\expandafter\def\csname cpp:TABU_LEN\endcsname{5}
\expandafter\def\csname cpp:TABU_VAL\endcsname{[7, 7, 7, 2, 1]}
\expandafter\def\csname cpp:TABU\endcsname{{.len=5, .val=malloc(5 * sizeof(int))}}
\expandafter\def\csname cpp:V\endcsname{nickels}
\expandafter\def\csname cpp:TABV_LEN\endcsname{8}
\expandafter\def\csname cpp:TABV_VAL\endcsname{[9, 1, 3, 8, 5, 1, 7, 8]}
\expandafter\def\csname cpp:TABV\endcsname{{.len=8, .val=malloc(8 * sizeof(int))}}

\def\C#1{\csname cpp:#1\endcsname}

\element{COMP2}{
  \begin{questionmult}{C1}
    What does function \C{FORALLIN} computes?
    \AMCBoxedAnswers
    \begin{reponses}
      \wrongchoice{the smallest value that is in \texttt{news} but not in \texttt{dislike}}
      \wrongchoice{the largest value that is in \texttt{news} but not in \texttt{dislike}}
      \wrongchoice{the sum of the values that are in \texttt{news} but not in \texttt{dislike}}
      \wrongchoice{the number of values that are in \texttt{news} but not in \texttt{dislike}}
      \wrongchoice{the smallest value that is both in \texttt{news} and in \texttt{dislike}}
      \wrongchoice{the largest value that is both in \texttt{news} and in \texttt{dislike}}
      \wrongchoice{the sum of the values that are both in \texttt{news} and in \texttt{dislike}}
      \wrongchoice{the number of values that are both in \texttt{news} and in \texttt{dislike}}
      \wrongchoice{the smallest value that is in \texttt{dislike} but not in \texttt{news}}
      \correctchoice{the largest value that is in \texttt{dislike} but not in \texttt{news}}
      \wrongchoice{the sum of the values that are in \texttt{dislike} but not in \texttt{news}}
      \wrongchoice{the number of values that are in \texttt{dislike} but not in \texttt{news}}
      \wrongchoice{the smallest value that is both in \texttt{dislike} and in \texttt{news}}
      \wrongchoice{the largest value that is both in \texttt{dislike} and in \texttt{news}}
      \wrongchoice{the sum of the values that are both in \texttt{dislike} and in \texttt{news}}
      \wrongchoice{the number of values that are both in \texttt{dislike} and in \texttt{news}}
    \end{reponses}
  \end{questionmult}
  \begin{questionmult}{C2}
    What does the program prints?
    \begin{multicols}{4}
      \AMCBoxedAnswers
      \begin{reponses}
        \wrongchoice{1}
        \wrongchoice{2}
        \wrongchoice{3}
        \wrongchoice{4}
        \wrongchoice{33}
        \wrongchoice{5}
        \wrongchoice{7}
        \correctchoice{9}
        \wrongchoice{11}
        \wrongchoice{14}
        \wrongchoice{16}
        \wrongchoice{22}
      \end{reponses}
    \end{multicols}
  \end{questionmult}
  \begin{questionmult}{C3}
    What does the program prints?
    \AMCnumericChoices{9}{digits=3,vertical=true,scoreexact=1,scoreapprox=0,borderwidth=1pt,sign=false}
  \end{questionmult}
}

